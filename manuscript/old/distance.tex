\documentclass[12pt,letterpaper]{tufte-handout}

\usepackage[english]{babel}

% Mathematical Notation
\usepackage{amsmath,amstext,amssymb}

\DeclareMathOperator*{\argmin}{arg\,min}
\DeclareMathOperator*{\argmax}{arg\,max}

\renewcommand{\Pr}{\mathsf{P}}
\newcommand{\prob}[1]{\Pr\left(#1\right)}
\newcommand{\given}{\mid}
\newcommand{\me}{\mathrm{e}}
\renewcommand{\emptyset}{\varnothing} % use a circle instead of a zero for the empty set, requires amssymb

% Lists
\usepackage{enumitem}
% usage: \begin{enumerate}[resume] will continue numbering from previous enumerate block

% Citations
%\usepackage{natbib}

\begin{document}
\section{Notes on a mean phylogenetic tree}

\subsection{Background}

We desire a method to define and calculate the mean of a sample of phylogenetic trees.
Many others have done recent work on the Fr\'{e}chet mean of phylogenetic trees
using a distance defined as the minimal path distance between two phylogenetic trees in tree space (xxx references).
The Fr\'{e}chet mean tree $a$ is defined as
$$
a = \argmin_{t \in {\cal T}} \sum_{i=1}^n d^2(t_i,t)
$$
where $t_1,\ldots,t_n$ is a sample of trees and ${\cal T}$ is the set of all phylogenetic trees
for some distance $d$.

Each phylogenetic tree $t$ is defined by its set of splits $t_S = \{s \in {\cal S} : s \in t\}$ and
the edge lengths associated with each split, $t(s)$
where ${\cal S}$ is the set of all possible splits.
Extend the definition so that $t(s) = 0$ if $s \notin t$.
Then define the distance between two trees $t_1$ and $t_2$ as
$$
d(t_1,t_2) = \sqrt{ \sum_{s \in {\cal S}} (t_1(s) - t_2(s))^2 }
$$
I do not know if this is equivalent to the path distance between trees or not.

With this definition of pairwise distance,
let
$$
D^2(a) = \sum_{i=1}^n d^2(t_i,a)
$$
be the sum of squared distances between a phylogenetic tree $a$ and sample trees $t_1,\ldots,t_n$.
A tree which minimizes $D^2$ is a Fr\'{e}chet mean tree.

Note the following straightforward derivation.
For simplicity, let $\sum_i$ represent $\sum_{i=1}^n$ and $\sum_s$ represent $\sum_{s \in {\cal S}}$.
\begin{eqnarray*}
D^2(a) & = & \sum_i \sum_s (t_i(s) - a(s))^2 \\
& = & \sum_i \sum_s (t_i(s))^2 - 2  \sum_s a(s) \sum_i t_i(s) + n \sum_s (a(s))^2 \\
& = & n \Big( \frac{\sum_i \sum_s (t_i(s))^2}{n} - 2  \sum_s a(s) \frac{\sum_i t_i(s)}{n} + \sum_s (a(s))^2 \Big)
\end{eqnarray*}
The first term is constant in $a$,
and so we can eliminate it when seeking a minimizer.
Define $\bar{s} = \sum_i t_i(s)/n$ to be the mean length of edges corresponding to the split $s$ over the sample,
treating trees without the split as zero values when computing the mean.
With this new notation, we seek to minimize
$$
\sum_s (a(s))^2 - 2 \sum_s a(s) \bar{s}
$$
As $\sum_s \bar{s}^2$ does not vary with $a$,
we can add it to the previous expression and we see that the Fr\'{e}chet mean minimizes
$$
\sum_s (a(s))^2 - 2 \sum_s a(s) \bar{s} + \sum_s \bar{s}^2 = \sum_s (a(s) - \bar{s})^2
$$
Decompose this sum over the splits in $a$ and those not in $a$.
$$
\sum_{s \in a} (a(s) - \bar{s})^2 + \sum_{s \notin a} (a(s) - \bar{s})^2
$$
The first term will be zero if the edges corresponding to splits $s \in a$ are given lengths $a(s) = \bar{s}$
and the second sum simplifies to $\sum_{s \notin a} \bar{s}^2$
as $a(s) = 0$ for all terms.
To minimize this sum,
we seek the tree $a$ with the set of compatible splits that maximizes the sum of squared mean samples splits.
Thus, the Fr\'{e}chet mean tree is
$$
a = \argmax_{ \overset{t \in {\cal T}}{t(s) = \bar{s}\ \text{for all}\ s \in t}} \sum_{s \in t} \bar{s}^2
$$

\nobibliography{distance}
\bibliographystyle{mbe}

\end{document}

